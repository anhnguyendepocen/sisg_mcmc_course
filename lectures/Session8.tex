\documentclass[pdftex]{article}
\usepackage[pdftex]{graphics}
\usepackage{subfigure}
\usepackage{hhline}
\usepackage[usenames,dvipsnames]{color}
\usepackage{colortbl}
\usepackage[screen,pdftex]{mcdlecture}
\newcommand{\bs}{\relax}
\newcommand{\es}{\newpage}
\fboxsep=.01\textwidth \fboxrule=1pt
\newsavebox{\savepar}
\newenvironment{boxit}{\begin{lrbox}{\savepar}
    \begin{minipage}[b]{0.975\textwidth}}
    {\end{minipage}\end{lrbox}\framebox{\usebox{\savepar}}}


%%%%%%%%%%%%%%%%%%%%%%%%%%%%%%%%%%%%%%%%%%%%%%%%%%%%%%%%%%
%% THE FOLLOWING ARE THINGS THAT WE MIGHT CHANGE FROM YEAR TO YEAR OR
%% VENUE TO VENUE
    \lhead{MCMC in Statistical Genetics}
    \lfoot{Dr Eric C. Anderson and Dr Matthew Stephens}
%	\lfoot{Dr Eric C. Anderson and Dr John Novembre}
%    \rfoot{UW - Summer Institute, July 2013}
%	 \rfoot{Edinburgh - European Institute, June 2012}
\rfoot{Brazil - Summer Institute, February 2014}

% on this one, be sure to update the venue and the module number
%\newcommand{\coursetitlepage}{European Institute in Statistical Genetics
%\newcommand{\coursetitlepage}{Summer Institute in Statistical Genetics
\newcommand{\coursetitlepage}{Brazilian Edition of the Summer \\Institute in Statistical Genetics

Module 9:

MCMC for Genetics}

%% Then update the schedule.  Note that I have broken that
%% out into a separate file like: schedule_table_edinburgh2012.tex
%% which is input in Overview.tex

%% Then be sure to change any time-sensitive events in the 
%% probability discussion in Matthew's first lecture.

%% And also update "structure_fun" link to my wiki to the right
%% year and venue.
%%%%%%%%%%%%%%%%%%%%%%%%%%%%%%%%%%%%%%%%%%%%%%%%%%%%%%%%%%


\begin{document}

\DeclareGraphicsExtensions{.jpg,.pdf,.png}%



%% Eric added a few things:
% some commands that Eric made for making a title while starting
% a new lecture and for making titles of new slides.
\newcommand{\newlecture}[1]{\newpage\begin{center}\section*{#1}\end{center}}
\newcommand{\newslide}[1]{\newpage\subsection*{#1: \hfil}}
 \newcommand{\Exp}{\Bbb{E}}
 \newcommand{\Var}{{\mathrm{Var}}}
 %% Some pretty etc.'s, etc...
\newcommand{\cf}{{\em cf.}}
\newcommand{\eg}{{\em e.g.},}
\newcommand{\ie}{{\em i.e.},}
\newcommand{\etal}{{\em et al.}\ }
\newcommand{\etc}{{\em etc.}}

%% some handy things for making bold math
\def\bm#1{\mathpalette\bmstyle{#1}}
\def\bmstyle#1#2{\mbox{\boldmath$#1#2$}}
\newcommand{\thh}{^\mathrm{th}}
\newcommand{\bpi}{{\pi}}
\newcommand{\mP}{\mathbf{P}}
\rhead{Session 8 - \thepage}

\section*{\hfil Practical on haplotype inference\hfil}

All input files are available at 

\url{http://stephenslab.uchicago.edu/sisg/}

You will need to save the input files example1.txt, example2.txt 
and example3.txt to your local computer in the same directory
as PHASE lives.

Then open an MSDOS window, and change to the directory where
PHASE lives.

\es\bs

\subsection*{Example 1}

The input file example1.txt contains an input file for the PHASE software.
Here is what it looks like, with some comments
\begin{verbatim}
3 (this is the number of individuals)
5 (number of SNPs)
P 100 200 300 400 500 (positions of SNPs) 
SSSSS (types of loci: all 5 are SNPs)
#1   (individual label - this is data for individual 1)
11111   (5 columns, 2 rows, each column is 1 genotype)
11111   (this case is homozygous for 1 allele at all SNPs)
#2
00000
00000
#3
00000  (this one is heterozygous at all SNPs)
11111
\end{verbatim}


\es\bs


Before running PHASE on this file, here are some things to think about:

\begin{itemize}
\item How many of the individuals have ambiguous haplotypes? 
\item What would you guess for the haplotypes of the ambiguous individual(s)?
\item How confident would you be? Very confident? Less confident? (What might
make you more confident?) 
\end{itemize}
\es\bs

\begin{itemize}
\item Try running PHASE on this file using

{\tt PHASE example1.txt example1.out}
 
Take a look at the output files example1.out, and example1.out\_pairs,
and see if PHASE's answers correspond to your intuition. 

You might also like to look at the output file example1.out\_freqs which
contains estimates of the population haplotype frequencies.


\item Try running PHASE again: 

{\tt PHASE example1.txt example1.2.out}

Do you get the same answers? You should, because by default
PHASE uses the same random numbers to simulate the Markov chain.

\end{itemize}

\es\bs

\subsection*{Changing the Seed}

To change the random numbers used, you need to use the -S option
to set the "seed":

PHASE -S332554 example1.txt example1.3.out

The answers should look similar, but not identical.

Performing multiple runs with different seeds is a helpful way to
check that you are running the algorithm long enough to get
reliable results.

\es\bs

\subsection*{Example 2}

The file example2.txt contains another small input file. This
one was created by putting together individuals by randomly
pairing haplotypes taken from a pool containing equal numbers
of the 8 possible 3-SNP haplotypes.

\begin{itemize}
\item What would you expect to happen for this input file?
\item Try running PHASE a few times, with different seeds, and compare the results.
\item As well as estimating the haplotypes for each individual (eg in the \_pairs file)
PHASE also estimates population haplotype frequencies: see the \_freqs file.
Compare these estimates with the description of how
the input file was created.
\item One way to estimate haplotype frequencies is to first estimate the haplotypes
for each individual and then to count up how many times each haplotype occurs
in these estimates. But this is the kind of 2-stage procedure that should be
avoided. Can you think of another way?
\item By default PHASE estimates recombination rates, and uses conditional
distributions based on these rates within a model known as the coalescent.
One can also specify, among other things, that PHASE is to use a Dirichlet
prior for the haplotype frequencies (use the -ME option), or to assume a coalescent prior with no
recombination (-MS). The main reason one might want to use these options instead
of the default is that for big problems they can run appreciably quicker.
Try running PHASE with these options on this data set, 
and compare the results (eg the \_freqs file) with the results from the default. 
\end{itemize}


\subsection*{Example 3}

In addition to estimating haplotypes, PHASE can also be used
to estimate recombination rates in a region, and to assess whether
the region contains a recombination "hotspot". By invoking the -MR1 1
option, PHASE assumes that the recombination rate in the region
is constant, with the possible exception of a single recombination hotspot
somewhere in the gene. PHASE makes various assumptions about the
location, width and intensity of the hotspot by making assumptions
about the prior distributions of these quantities. It then outputs
a \_hotspot file, which contains a sample from the posterior distribution
for these quantities.


The input file example3.txt
contains data from a gene CD36.
Run PHASE using the -MR1 1 option:

./PHASE -MR1 1 example3.txt example3.out

and examine the contents of the file example3.out\_hotspot.

\es\bs

The following code can be used to read the \_hotspot file into R and plot summaries.
(Or you could use Excel if you prefer.) Try performing multiple runs
of PHASE and comparing results (eg the posterior distribution of the hotspot
intensity) for different seeds.

\begin{verbatim}
h = read.table("example3.out_hotspot")
hist(h[,4]) # h[,4] contains samples from the posterior 
            # distribution of intensities
plot(h[,4])
mean(h[,4]>1) # returns the proportion of samples 
              # where the intensity is > 1
              # (note that intensity 1 corresponds to 
              # "no hotspot")
hist(h[,3]-h[,2]) # h[,3] and h[,2] are samples from the 
                  # limits of the hotspot
plot(h[,3]-h[,2])
hist(h[,1]) # h[,1] is an estimate of the recombination
            # parameter outside the hotspot
sum(h[,4]>1)
\end{verbatim}


\end{document}
